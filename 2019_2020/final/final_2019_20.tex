\documentclass[12pt]{article}

\usepackage{tikz} % картинки в tikz
\usepackage{microtype} % свешивание пунктуации

\usepackage{array} % для столбцов фиксированной ширины

\usepackage{indentfirst} % отступ в первом параграфе

\usepackage{sectsty} % для центрирования названий частей
\allsectionsfont{\centering}

\usepackage{amsmath, amssymb} % куча стандартных математических плюшек


\usepackage{comment}

\usepackage[top=2cm, left=1.2cm, right=1.2cm, bottom=2cm]{geometry} % размер текста на странице

\usepackage{lastpage} % чтобы узнать номер последней страницы

\usepackage{enumitem} % дополнительные плюшки для списков
%  например \begin{enumerate}[resume] позволяет продолжить нумерацию в новом списке
\usepackage{caption}


\usepackage{fancyhdr} % весёлые колонтитулы
\pagestyle{fancy}
\lhead{Time series}
\chead{}
\rhead{2020-06-26, final exam}
\lfoot{}
\cfoot{}
\rfoot{\thepage/\pageref{LastPage}}
\renewcommand{\headrulewidth}{0.4pt}
\renewcommand{\footrulewidth}{0.4pt}


\let\P\relax
\DeclareMathOperator{\P}{\mathbb{P}}

\usepackage{todonotes} % для вставки в документ заметок о том, что осталось сделать
% \todo{Здесь надо коэффициенты исправить}
% \missingfigure{Здесь будет Последний день Помпеи}
% \listoftodos --- печатает все поставленные \todo'шки


% более красивые таблицы
\usepackage{booktabs}
% заповеди из докупентации:
% 1. Не используйте вертикальные линни
% 2. Не используйте двойные линии
% 3. Единицы измерения - в шапку таблицы
% 4. Не сокращайте .1 вместо 0.1
% 5. Повторяющееся значение повторяйте, а не говорите "то же"



\usepackage{fontspec}
\usepackage{polyglossia}

\setmainlanguage{russian}
\setotherlanguages{english}

% download "Linux Libertine" fonts:
% http://www.linuxlibertine.org/index.php?id=91&L=1
\setmainfont{Linux Libertine O} % or Helvetica, Arial, Cambria
% why do we need \newfontfamily:
% http://tex.stackexchange.com/questions/91507/
\newfontfamily{\cyrillicfonttt}{Linux Libertine O}

\AddEnumerateCounter{\asbuk}{\russian@alph}{щ} % для списков с русскими буквами
%\setlist[enumerate, 2]{label=\asbuk*),ref=\asbuk*}

%% эконометрические сокращения
\DeclareMathOperator{\Cov}{Cov}
\DeclareMathOperator{\Corr}{Corr}
\DeclareMathOperator{\Var}{Var}
\DeclareMathOperator{\E}{E}
\def \hb{\hat{\beta}}
\def \hs{\hat{\sigma}}
\def \htheta{\hat{\theta}}
\def \s{\sigma}
\def \hy{\hat{y}}
\def \hY{\hat{Y}}
\def \v1{\vec{1}}
\def \e{\varepsilon}
\def \he{\hat{\e}}
\def \z{z}
\def \hVar{\widehat{\Var}}
\def \hCorr{\widehat{\Corr}}
\def \hCov{\widehat{\Cov}}
\def \cN{\mathcal{N}}


\begin{document}

\section*{Promo-code activation :)}

You have two options! 

You can try to solve all the proposed problems and submit your solutions as a pdf file.
In this case I will honestly grade them.

You can write a promo-code \#arma\_dillo. 
In this case I will ignore your solutions and grade your exam as 4/10. 

\section*{Problems}

\begin{enumerate}

\item Consider a stationary solution of the equation $y_t = 2 + 0.7y_{t-1} + u_t - 0.5u_{t-1}$, 
	where $u_t$ is a white noise process. 

\begin{enumerate}
	\item Calculate $\E(y_t)$, first two values of autocorrelation and partial autocorrelation functions.
	\item Assuming normality and independence of $u_t$, $u_t \sim \cN(0; 9)$, $u_{100}=-1$, $y_{100}=5$, 
	calculate short-term 95\% predictive interval for $y_{101}$ and long-term 95\% predictive interval for $y_{100+h}$ 
	where $h \to \infty$.
\end{enumerate}
	


\item Consider a stationary solution of the equation $y_t = 2 + 0.7y_{t-1} -0.12y_{t-2} + u_t$, 
where $u_t$ is a white noise process. 

James Bond assumes the wrong model $y_t = \beta_1 + \beta_2 y_{t-1} + u_t$ 
and estimate the regression $\hat y_t = \hat\beta_1 + \hat\beta_2 y_{t-1}$ by OLS.

\begin{enumerate}
	\item Find the probability limit of $\hat\beta_1$ and $\hat\beta_2$. 
	\item Using $\hat\beta_1$, $\hat\beta_2$ and his wrong assumption James Bond tries to estimate $\mu = \E(y_t)$. 
	Find the probability limit of this estimator $\hat\mu$. Will it be consistent?
\end{enumerate}

\item Consider a $ETS(AAA)$ model for monthly data described by the system

\[
\begin{cases}
y_t = \ell_{t-1} + b_{t-1} + s_{t-12} +  \varepsilon_t \\
\ell_t = \ell_{t-1} + b_{t-1} + \alpha \varepsilon_t \\
b_t = b_{t-1} + \beta \varepsilon_t \\
s_t = s_{t-12} + \gamma \varepsilon_t \\
\end{cases}
\]

Represent it as ARIMA model. Find all the coefficients of this $ARIMA$ representation.


\newpage
\item Consider two independent $MA(1)$ process with zero expected value and unit variance, $A_t$ and $B_t$.
The first values of autocorrelation function are $\rho_1^A = 0.5$ and $\rho_1^B=0.5$.

The process $S_t$ is just the sum of these two, $S_t = A_t + B_t$.

\begin{enumerate}
	\item Classify $S_t$ as $ARIMA(p, d, q)$ process. Find $p$, $d$ and $q$.
	\item Calculate the autocorrelation function for $S_t$. 
\end{enumerate}

\item Consider the process $y_t = u_1 \sin t + u_2 \cos t$, where $u_t$ — is a white noise.

\begin{enumerate}
	\item Is $y_t$ stationary?
	\item Can $y_t$ be represented as $ARIMA(p, d, q)$ process? Find $p$, $d$, $q$ if possible.
\end{enumerate}

Hint: $\cos(a+b) = \cos a \cos b - \sin a \sin b$, $\sin (a + b) = \sin a \cos b + \sin b \cos a$. 


\item Variables $(x_t)$ are independent and are equal to $0$ or $1$ with equal probability, 
$u_t$ are independent $\cN(0; 1)$. Consider the process $z_t = x_t (1-x_{t-1}) u_t$.

\begin{enumerate}
	\item Is $z_t$ a stationary process? Find its autocorrelation function if it is stationary.
	\item You know that $z_{100} = 2.3$. Find one and two step ahead point and interval forecasts.
	What is special about interval forecasts in this case?
\end{enumerate}


\end{enumerate}

\end{document}
