\documentclass[12pt]{article}

\usepackage{tikz} % картинки в tikz
\usepackage{microtype} % свешивание пунктуации

\usepackage{array} % для столбцов фиксированной ширины

\usepackage{indentfirst} % отступ в первом параграфе

\usepackage{sectsty} % для центрирования названий частей
\allsectionsfont{\centering}

\usepackage{amsmath, amssymb} % куча стандартных математических плюшек


\usepackage{comment}

\usepackage[top=2cm, left=1.2cm, right=1.2cm, bottom=2cm]{geometry} % размер текста на странице

\usepackage{lastpage} % чтобы узнать номер последней страницы

\usepackage{enumitem} % дополнительные плюшки для списков
%  например \begin{enumerate}[resume] позволяет продолжить нумерацию в новом списке
\usepackage{caption}


\usepackage{fancyhdr} % весёлые колонтитулы
\pagestyle{fancy}
\lhead{Time series}
\chead{}
\rhead{2021-03-29, final exam}
\lfoot{}
\cfoot{}
\rfoot{\thepage/\pageref{LastPage}}
\renewcommand{\headrulewidth}{0.4pt}
\renewcommand{\footrulewidth}{0.4pt}


\let\P\relax
\DeclareMathOperator{\P}{\mathbb{P}}

\usepackage{todonotes} % для вставки в документ заметок о том, что осталось сделать
% \todo{Здесь надо коэффициенты исправить}
% \missingfigure{Здесь будет Последний день Помпеи}
% \listoftodos --- печатает все поставленные \todo'шки


% более красивые таблицы
\usepackage{booktabs}
% заповеди из докупентации:
% 1. Не используйте вертикальные линни
% 2. Не используйте двойные линии
% 3. Единицы измерения - в шапку таблицы
% 4. Не сокращайте .1 вместо 0.1
% 5. Повторяющееся значение повторяйте, а не говорите "то же"



\usepackage{fontspec}
\usepackage{polyglossia}

\setmainlanguage{russian}
\setotherlanguages{english}

% download "Linux Libertine" fonts:
% http://www.linuxlibertine.org/index.php?id=91&L=1
\setmainfont{Linux Libertine O} % or Helvetica, Arial, Cambria
% why do we need \newfontfamily:
% http://tex.stackexchange.com/questions/91507/
\newfontfamily{\cyrillicfonttt}{Linux Libertine O}

\AddEnumerateCounter{\asbuk}{\russian@alph}{щ} % для списков с русскими буквами
%\setlist[enumerate, 2]{label=\asbuk*),ref=\asbuk*}

%% эконометрические сокращения
\DeclareMathOperator{\Cov}{Cov}
\DeclareMathOperator{\Corr}{Corr}
\DeclareMathOperator{\Var}{Var}
\DeclareMathOperator{\E}{E}
\def \hb{\hat{\beta}}
\def \hs{\hat{\sigma}}
\def \htheta{\hat{\theta}}
\def \s{\sigma}
\def \hy{\hat{y}}
\def \hY{\hat{Y}}
\def \v1{\vec{1}}
\def \e{\varepsilon}
\def \he{\hat{\e}}
\def \z{z}
\def \hVar{\widehat{\Var}}
\def \hCorr{\widehat{\Corr}}
\def \hCov{\widehat{\Cov}}
\def \cN{\mathcal{N}}


\begin{document}


\begin{enumerate}


\section*{Easy problems!}

\item The semi-annual $y_t$ is modelled by $ETS(AAA)$ process:
    
\[
\begin{cases}
	u_t \sim \cN(0; 16) \\
	s_t = s_{t-2} + 0.1 u_t \\
	b_t = b_{t-1} + 0.2 u_t \\
	\ell_t = \ell_{t-1} + b_{t-1} + 0.3 u_t \\
	y_t = \ell_{t-1} + b_{t-1} + s_{t-2} + u_t \\
\end{cases}    
\]

Given that $s_{100} = 2$, $s_{99} = -1.9$, $b_{100} = 0.5$, $\ell_{100} = 4$ find 95\% predictive interval for $y_{101}$ and $y_{102}$. 


\item Consider a stationary solution of the equation $y_t = 2 + y_{t-1} - 0.25 y_{t-2} + u_t $, 
	where $u_t$ is a white noise process. 

\begin{enumerate}
	\item Calculate $\E(y_t)$, the first two values of autocorrelation function.
	\item Assuming normality and independence of $u_t$, $u_t \sim \cN(0; 16)$, $u_{100}=-1$, $y_{100}=5$, 
	calculate short-term 95\% predictive interval for $y_{101}$ and long-term 95\% predictive interval for $y_{100+h}$ 
	where $h \to \infty$.
\end{enumerate}


\item Consider six observations $y = (10, 20, 40, 100, 110, 150)$. 
Construct regression tree to predict $y_t$ using $y_{t-1}$ as unique predictor.
The value of $TSS$ is used to split a node. Growing of tree stops when each node contains no more than 2 observations. 
To split two values use the mean between them. 

\begin{enumerate}
	\item Draw your tree!
	\item What is the forecast of this tree for the next observation?
\end{enumerate}



\section*{A little bit harder!}


\item Consider the $ETS(AAN)$ process:
    
\[
\begin{cases}
	u_t \sim \cN(0; \sigma^2) \\
	b_t = b_{t-1} + \beta u_t \\
	\ell_t = \ell_{t-1} + b_{t-1} + \alpha u_t \\
	y_t = \ell_{t-1} + b_{t-1} + u_t \\
\end{cases}    
\]
with initial constants $b_0$ and $\ell_0$. 

Find $\Cov(y_t, y_s)$. Is the $ETS(AAN)$ stationary?


\item Check whether the sum of two independent $ETS(ANN)$ processes is a $ETS(ANN)$ process.

\item Consider $MA(1)$ process $y_t = u_t + 2u_{t-1}$ where $(u_t)$ is a white noise with variance $\sigma$.


\begin{enumerate}
	\item Find $\Cov(y_t, y_s)$ for all $t$ and $s$.
	\item Find best linear prediction for $y_t$ using $y_{t-1}$ and $y_{t-2}$. 

\end{enumerate}

I will give you a hint for the point b. You need to find $\alpha_1$ and $\alpha_2$ in 
$y_t = \alpha_1 y_{t-1} + \alpha_2 y_{t-2} + w_t$ such that prediction error $w_t$ is uncorrelated with predictors $y_{t-1}$ and $y_{t-2}$.
\end{enumerate}

\end{document}
